\documentclass[14pt]{beamer}
% Možne velikosti pisav so: 8pt, 9pt, 10pt, 11pt, 12pt, 14pt, 17pt, 20pt

% Nastavitve za šumnike
\usepackage[T1]{fontenc}
\usepackage[utf8]{inputenc}
% \usepackage[slovene]{babel}

% To potrebujemo, da bomo lahko delali zapiske za predavatelja
\usepackage{pgfpages}

% Treba je ločiti zapiske za predavatelja in prosojnice za občinstvo.
% To lahko naredimo v beamerju, glej
% https://gist.github.com/andrejbauer/ac361549ac2186be0cdb

% Primer tako narejenih prosojnic:
% http://math.andrej.com/wp-content/uploads/2016/06/hott-reals-cca2016.pdf

% Izberemo, ali bomo naredili samo prosojnice, samo zapiske ali oboje

\setbeameroption{hide notes}                        % samo prosojnice
% \setbeameroption{show only notes}                   % samo zapiski
%\setbeameroption{show notes on second screen=right}  % oboje

% Ali naj se \item-i pojavljajo zaporedoma, ali vsi naenkrat?
% Naj se pojavljajo zaporedoma:
% \beamerdefaultoverlayspecification{<+->}

% Minimalistični still
% \mode<presentation>
 \usetheme{Boadilla}
 %\useoutertheme{default}
 \useoutertheme{default}
% \usecolortheme{rose}

% Malo manj dolgočasna pisava
\usepackage{palatino}
\usefonttheme{serif}

% Izklopimo navigacijske simbole, ker so neuporabni
\setbeamertemplate{navigation symbols}{}

% Nastavimo ozadje

% Aktiviramo oštevilčenje strani
%\setbeamertemplate{footline}[frame number]{}

% Stil za zapiske
%\setbeamertemplate{note page}{\pagecolor{yellow!5}\insertnote}

\usepackage{amsmath}
\usepackage{amssymb}

\usepackage{tikz-cd} 
\usetikzlibrary{babel} % this fixes problems with tikz-cd

\usepackage{bussproofs} % package for writing natural deduction tree-style proofs

\DeclareMathOperator{\graph}{graph}
\theoremstyle{plain}
\newtheorem{proposition}{Proposition}

\setbeamercolor{emph}{fg=red}
\renewcommand<>{\emph}[1]{%
  {\usebeamercolor[fg]{emph}\only#2{\itshape}#1}%
}

\title{Regular logic and regular categories}
\author{}
\date{}

\begin{document}

  \begin{frame}
    \titlepage

  \end{frame}


  % \begin{frame}
  %   \frametitle{High level idea}
  %   \pause
  %   \begin{itemize} 
  %     \item We have a logic $\mathcal{L}$ and a notion of a theory $T$ in this logic.
  %     \pause
  %     \item We define the concept of a model of the theory $T$ in categories with appropriate structure.
  %     \pause
  %     \item Soundness of the logic $\mathcal{L}$. Everything we can prove with the rules
  %       of inference from $\mathcal{L}$ must hold in all models.
  %     \pause
  %     \item Completeness. There is a special model, that allows us to show, that everything
  %       that holds for all models of $T$, can be proven in $T$.
  %   \end{itemize}
  % \end{frame}

  % \begin{frame}
  %   \frametitle{Gameplan}
  %   \begin{itemize}
  %     \item<1-> Define regular logic
  %     \item<2-> Define regular categories
  %     \item<3-> Examples
  %     \item<4-> Regular epimorphisms
  %     \item<5-> Interpretation \& model
  %     \item<6-> Soundness
  %     \item<7-> Category of models
  %     \item<8-> Internal logic
  %     \item<9-> Syntactic category
  %   \end{itemize}
  % \end{frame}

  % \begin{frame}
  %   \frametitle{Regular logic}
  %   \emph{Regular signature} $\Sigma$:
  %   \begin{itemize}
  %     \pause
  %     \item set of sorts $\mathrm{sort}_{\Sigma} = \{ X_{1}, X_{2}, X_{3}, \ldots{} \}$
  %     \pause
  %     \item set of function symbols $\mathrm{func}_{\Sigma}$ 
  %     \[ f : ( X_{1}, X_{2}, \ldots, X_{n} ) \to Y \]
  %     where $X_{1}, \ldots{}, X_{n}, Y$ are sorts.
  %     \pause
  %     \item set of relation symbols $\mathrm{rel}_{\Sigma}$ 
  %       \[ R : (X_{1}, X_{2}, \ldots, X_{n} ) \]
  %     where $X_{1}, \ldots{}, X_{n}$ are sorts.
  %     \end{itemize}
  % \end{frame}

  % \begin{frame}
  %   \begin{block}{Special case}
  %     Constants $c : () \to Y$ are just
  %     function symbols from the empty domain.
  %   \end{block}
  %   \pause 
  %   \begin{block}{Note}
  %     We frequently denote $X_1, \ldots, X_n$ as $\bar{X}$.
  %   \end{block}
  % \end{frame}

  % \begin{frame}
  %   Regular signature $\Sigma$ $\quad \rightsquigarrow \quad$ language $\mathcal{L}(\Sigma)$.
  %   \pause \\
  %   Consists of \emph{variables}, \emph{terms} and \emph{formulas}.
  %   \pause
  %   \begin{itemize}
  %     \pause
  %     \item For each sort $X$ we have countably many variables $x_1, x_2, x_3, \ldots : X$.
  %     \vspace{1em}
  %     \pause
  %     \item We define terms inductively:
  %       \begin{itemize}
  %         \vspace{1em}
  %         \pause
  %         \item Each variable $x : X$ is a term of sort $X$
  %         \vspace{1em}
  %         \pause
  %         \item Terms $t_1, \ldots t_n$ of sorts $X_1, \ldots X_n$
  %           and $f : (X_1, \ldots, X_n) \to Y$ function symbol, then $f(t_1, \ldots, t_n)$
  %           is a term of sort $Y$
  %       \end{itemize}
  %   \end{itemize}
  % \end{frame}

  % \begin{frame}
  %   \begin{itemize}
  %     \item We define formulas inductively:
  %     \vspace{1em}
  %     \pause
  %       \begin{itemize}
  %         \item $t_1$ and $t_2$ terms of sort $X$, then $t_1 =_X t_2$ is a formula.
  %           \pause
  %           \note{We usually don't write the index $X$ if it's clear from context}
  %         \vspace{1em}
  %         \pause
  %         \item The logical constant $\top$ is a formula. 
  %         \note{It represents a true statement}
  %         \vspace{1em}
  %         \pause
  %         \item Terms $t_1, \ldots t_n$ of sorts $X_1, \ldots X_n$
  %           and $R : (X_1, \ldots, X_n)$ relation symbol, then $R(t_1, \ldots, t_n)$
  %           is a formula.
  %         \vspace{1em}
  %         \pause
  %         \item $\varphi$ and $\psi$ formulas, then $\varphi \land \psi$ and $\exists (x : X) \varphi$
  %           are formulas.
  %       \end{itemize}
  %   \end{itemize}
  % \end{frame}

  % \begin{frame}
  %   \frametitle{Theory}
  %   A \emph{theory} $T$ formulated in the language $\mathcal{L}(\Sigma)$ consists of a set
  %   of sequents \[ \Gamma \mid \varphi \vdash \psi ,\] \pause where $\Gamma$ is
  %   a \emph{context} of variables, contains all the free variables in
  %   $\varphi$ and $\psi$.
  %   \pause \\
  %   \vspace{1em}
  %   We interpret this sequent as: "in the theory $T$, in the context $\Gamma$, $\varphi$ implies $\psi$."
  %   \pause \\
  %   \vspace{1em}
  %   If the premise of a sequent if $\top$, then we write it simply as $\Gamma \mid \psi$.
  %   \note{and say $\psi$ holds in the context $\Gamma$.}
  % \end{frame}

  % \begin{frame}
  %   \frametitle{Inference rules}
  %   \begin{prooftree}
  %     \AxiomC{$\Gamma_1 \mid A_1$} \AxiomC{$\ldots$} \AxiomC{$\Gamma_n \mid A_n$}
  %     \TrinaryInfC{$\Delta \mid B$}
  %   \end{prooftree}
  %   \vspace{3em}
  %   where $\Gamma_i \mid A_i$ are sequents of the form $\Gamma \mid \varphi_i \vdash \psi_i$.
  %   We divide them into $3$ categories.
    
  % \end{frame}

  % \begin{frame}
  %   \frametitle{Structural rules}
  %     \begin{enumerate}
  %     \item
  %       \begin{prooftree}
  %         \AxiomC{} \UnaryInfC{$\Gamma \mid \varphi \vdash \varphi$}
  %       \end{prooftree}
  %       \note[item]{reflexivity}
  %     \item
  %       \begin{prooftree}
  %         \AxiomC{$\Gamma \mid \varphi \vdash \psi$} \AxiomC{$\Gamma \mid \psi \vdash \rho$}
  %         \BinaryInfC{$\Gamma \mid \varphi \vdash r$}
  %       \end{prooftree}
  %       \note[item]{transitivity}
  %     \item
  %       \begin{prooftree}
  %         \AxiomC{$\Gamma \mid \varphi \vdash \psi$}
  %         \UnaryInfC{$\Gamma, y:Y \mid \varphi \vdash \psi$}
  %       \end{prooftree}
  %       \note[item]{superflous variable}
  %     \item
  %       \begin{prooftree}
  %         \AxiomC{$\Gamma, y:B \mid \varphi \vdash \psi$}
  %         \AxiomC{$\Gamma \mid b : B$}
  %         \BinaryInfC{$\Gamma \mid \varphi[b/y] \vdash \psi[b/y]$}
  %       \end{prooftree}
  %       \note[item]{substitution}
  %     \end{enumerate}

  % \end{frame}


  % \begin{frame}
  % \frametitle{Rules of logic}

  %     \begin{enumerate}\addtolength{\itemsep}{0.7\baselineskip}
  %   \item
  %     \begin{prooftree}
  %       \AxiomC{} \UnaryInfC{$\Gamma \mid \varphi \vdash \top$}
  %     \end{prooftree}
  %   \item
  %     \begin{prooftree}
  %       \AxiomC{$\Gamma \mid \rho \vdash \varphi$} \AxiomC{$\Gamma \mid \rho \vdash \psi$} \doubleLine
  %       \BinaryInfC{$\Gamma \mid \rho \vdash \varphi \wedge \psi$}
  %     \end{prooftree}
  %   \item
  %     \begin{prooftree}
  %       \AxiomC{$\Gamma \mid \exists (y:Y) \psi \vdash \varphi$} \doubleLine
  %       \UnaryInfC{$\Gamma, y:Y \mid \psi \vdash \varphi$}
  %     \end{prooftree}
  %   \end{enumerate}
  % \end{frame}

  % \begin{frame}
  %   \frametitle{Rules of equality}
    
  %   \begin{enumerate}\addtolength{\itemsep}{0.7\baselineskip}
  %   \item $x:X \mid \top \vdash x = x$
  %   \item
  %     $x_1:X,x_2:X \mid x_1 = x_2 \vdash x_2 = x_1$
  %   \item
  %     $x_1,x_2,x_3:X \mid x_1 = x_2 \wedge x_2 = x_3 \vdash x_1 = x_3$
  %   \item
  %     $\bar{x},\bar{y}:\bar{X} \mid \bar{x} = \bar{y} \vdash
  %     f(\bar{x}) = f(\bar{y})$
  %   \item
  %     $\bar{x},\bar{y} : \bar{X} \mid \bar{x} = \bar{y} \wedge R(\bar{x}) \vdash
  %     \mathrm{R}(\bar{y})$, where
  %     $R : \bar{X}$
  %   \end{enumerate}

  % \end{frame}

  % \begin{frame}
  %   \frametitle{Deduction relation}
  %   We define a deduction in $T$, which we denote as 
  %   \[\Gamma \mid \varphi \vdash_T \psi\]
  %   \pause
  %   is a derivation with the above rules, with an added axiom
  %   \[  \mathrm{FV}(\varphi) \cup \mathrm{FV}(\psi) \mid \varphi \vdash_T \psi \]
  %   for each sequent $\mathrm{FV}(\varphi) \cup \mathrm{FV}(\psi)  \mid \varphi \vdash \psi$ in $T$.
  % \end{frame}

  % \begin{frame}[fragile]
  %   \frametitle{Example}
  %   We prove 
  %   \begin{prooftree}
  %     \AxiomC{$\Gamma \mid \rho \vdash \psi(t)$} \AxiomC{$\Gamma \mid t : T$}
  %     \BinaryInfC{$\Gamma \mid \rho \vdash \exists (x:X) \psi(x)$}
  %   \end{prooftree}
  %   \pause
  %   via
  %   \begin{prooftree}
  %     \footnotesize
  %     \AxiomC{$\Gamma \mid \rho \vdash \psi(t)$} \AxiomC{}
  %     \RightLabel{\scriptsize(1.1)}
  %     \UnaryInfC{\(\Gamma \mid \exists (x:X) \psi(x) \vdash \exists (x:X) \psi(x)\)}
  %     \RightLabel{\scriptsize(2.3)}
  %     \UnaryInfC{$\Gamma, x:X \mid \psi(x) \vdash \exists (x:X) \psi(x)$}
  %     \AxiomC{$\Gamma \mid t:T$}
  %     \RightLabel{\scriptsize(1.4)}
  %     \BinaryInfC{$\Gamma \mid \psi(t) \vdash \exists (x:X) \psi(x)$}
  %     \RightLabel{\scriptsize(1.2)}
  %     \BinaryInfC{$\Gamma \mid \rho \vdash \exists (x:X) \psi(x)$}
  %   \end{prooftree}
  % \end{frame}

  % \begin{frame}[fragile]
  %   \frametitle{Regular categories}
  %   To define a model, we need an appropriate class of categories.
  %   \pause
  %   \note[item]{First we need an algebra of subobjects}
  %   \begin{definition}
  %     Let $\mathcal{C}$ be a category and $X \in \mathcal{C}$.
  %     A \emph{subobject} of $X$ is an equivalence class of monomorphisms
  %     $\alpha : A \rightarrowtail X$. 
  %     \pause
  %     % Two monomorphisms $\alpha : A \to X$ and
  %     % $\beta : B \to X$ are equivalent if they are isomorphic over $X$.
      
  %     \begin{center}
  %       \begin{tikzcd}
  %         A \ar[dr, "\alpha"'] \ar[rr, "\cong"] & & B \ar[dl, "\beta"] \\
  %         & X &
  %       \end{tikzcd}
  %     \end{center}
  %   \end{definition}
  % \end{frame}

  % \begin{frame}[fragile]
  %   \begin{definition}
  %     The class of all subobjects of $X$ denoted as $\mathrm{Sub}(X)$.
  %   \end{definition}
  %   \pause
  %   Equip $\mathrm{Sub}(X)$ with a partial order, where $A \leq B$ if
  %   $\alpha : A \rightarrowtail X$ factors through $\beta : B \rightarrowtail X$.
  %   \pause
  %   \begin{equation*}
  %     \begin{tikzcd}
  %       A \ar[dr, "\alpha"'] \ar[rr] & & B \ar[dl, "\beta"] \\
  %       & X &
  %     \end{tikzcd}
  %   \end{equation*}
  %   \note[item]{the arrow $A \to B$ is a monomorphism}
  % \end{frame}

  % \begin{frame}[fragile]
  %   Let's say $\mathcal{C}$ has all pullbacks.
  %   \pause
  %   Then for any $A, B \in \mathrm{Sub}(X)$ we can define their \emph{meet} or
  %   \emph{greatest lower bound} as the pullback
  %   \pause
  %   \begin{center}
  %     \begin{tikzcd}
  %       A \times_X B \ar[d] \ar[r] & B \ar[d, "\beta"] \\
  %       A \ar[r, "\alpha"] & X
  %     \end{tikzcd}
  %   \end{center}
  %   \pause
  %   This makes $\mathrm{Sub}(X)$ into a $\land$-semilattice.
  % \end{frame}

  % \begin{frame}[fragile]
  %   Let $f : X \to Y$ be a morphism in $\mathcal{C}$. We get a map
  %   \[ f^{-1} : Sub(Y) \to Sub(X) \]
  %   \pause
  %   induced by the pullback over $f$
  %   \begin{center}
  %     \begin{tikzcd}
  %       f^{-1}B \ar[d] \ar[r] & B \ar[d, "\beta"] \\
  %       X \ar[r, "f"] & Y
  %     \end{tikzcd}
  %   \end{center}
  %   \pause
  %   This map preserves meets.
  % \end{frame}

  % \begin{frame}[fragile]
  %   \begin{definition}
  %     Let $f : X \to Y$ be a morphism in $\mathcal{C}$. We call the pair of monomorphisms
  %     $(p_1, p_2)$ in the pullback
  %     \begin{center}
  %       \begin{tikzcd}
  %         X \times_Y X \ar[d, "p_1"'] \ar[r, "p_2"] & X \ar[d, "f"] \\
  %         X \ar[r, "f"] & Y
  %       \end{tikzcd}
  %     \end{center}
  %     the \emph{kernel pair} of $f$.
  %   \end{definition}
  % \end{frame}

  % \begin{frame}[fragile]
  %   \begin{definition}
  %     An epimorphism $f : X \to Y$ is called \emph{regular} if it is the coequaliser
  %     of some pair of morphisms.
  %     \begin{center}
  %       \begin{tikzcd}
  %         Z \ar[r, shift left, "g"] \ar[r, shift right, "h"'] & X \ar[r, "f"] & Y 
  %       \end{tikzcd}
  %     \end{center}
  %   \end{definition}
  %   \pause
  %   \begin{definition}
  %     A category $\mathcal{C}$ is \emph{regular} \pause if it has all finite limits,
  %     \pause
  %     each kernel pair has a coequaliser \pause and pullbacks preserve regular epimorphisms.
  %   \end{definition}
  % \end{frame}

  % \begin{frame}
  %   \begin{definition}
  %     Let $\mathcal{C}$ and $\mathcal{D}$ be regular categories.A functor
  %     $F : \mathcal{C} \to \mathcal{D}$ is called \emph{regular} if it preserves finite
  %     limits and coequalisers of kernel pairs.
  %   \end{definition}
  %   \pause
  %   \begin{definition}
  %     $\mathbf{RegCat}$ is the category whose objects are regular categories and
  %     morphims are regualar functors.
  %   \end{definition}
  % \end{frame}

  % \begin{frame}
  %   \frametitle{Examples}
  %   \begin{itemize}
  %     \item<1-> The category $\mathbf{Set}$ is regular.
  %     \item<2-> $\mathbf{Grp}$ is regular.
  %     \item<3-> If $\mathbb{T}$ is an algebraic theory and $\mathcal{C}$ is regular,
  %       then $\mathbf{Mod}(\mathbb{T}, \mathcal{C})$ is regular.
  %     \item<4-> If $\mathcal{C}$ is regular and $X \in \mathcal{C}$ then $\mathcal{C}/X$
  %       is regular.
  %     \item<5-> $\mathbf{Top}^{\mathrm{op}}$ is regular.
  %   \end{itemize}
  % \end{frame}

  % \begin{frame}[fragile]
  %   \frametitle{Grp}
  %   Pullbacks in $\mathbf{Grp}$ are constructed just like in $\mathbf{Set}$
  %   \begin{equation*}
  %     \begin{tikzcd}
  %       G \times_K H \ar[d, "p_1"'] \ar[r, "p_2"] & H \ar[d, "h"] \\
  %       G \ar[r, "g"] & K
  %     \end{tikzcd}
  %   \end{equation*}
  %   \pause
  %   $$G \times_K H = \{(x,y) \in G \times H \mid g(x) = h(y) \}$$ 
  %   $$p_1(x,y) = x, \quad p_2(x,y) = y$$
  %   \pause
  %   The terminal object in $\mathbf{Grp}$ is the group $G = \{ \ast \}$
  % \end{frame}

  % \begin{frame}
  %   \begin{proposition}
  %     In $\mathbf{Grp}$ every epimorphism is regular and every epimorphism is a coequalizer.
  %   \end{proposition}
  %   \vspace{1em}
  %   The coequaliser of a kernel pair $(p_1, p_2)$ of the homomorphism $f : G \to H$
  %   is just a quotient map.
  % \end{frame}
  
  % \begin{frame}[fragile]
  %   \frametitle{Grp}
  %   Show that pullbacks preserve regular epimorphisms.
  %   \begin{equation*}
  %     \begin{tikzcd}
  %       G \times_K H \ar[d, "p_1"'] \ar[r, "p_2"] & H \ar[d, two heads, "h"] \\
  %       G \ar[r, "g"] & K
  %     \end{tikzcd}
  %   \end{equation*}
  %   \pause
  %   Let $x \in G$. Can we find $(x,y) \in G \times_K H$ such that $p_1(x,y) = x$?
  %   \pause
  %   Let $y := g(x)$. Then $\exists y' \in H$ s.t. $y = h(y')$.
  %   \pause
  %   Then $g(x) = h(y')$, hence $(x,y') \in G \times_K H$ and $p_1(x,y') = x$.
    
  % \end{frame}

  % \begin{frame}
  %   \frametitle{Regular epimorphisms $\approx$ surjections}
  %     Let $\mathcal{C}$ be a regular category, then:
  %     \begin{itemize}
  %       \item<2-> Each regular epimorphism is the coequaliser of its kernel pair.
  %       \item<3-> A morphism that is both a regular epimorphism and a monomorphism is an
  %         isomorphism.
  %       \item<4-> A composition of regular epimorphisms is a regular epimorphism.
  %     \end{itemize}
  % \end{frame}

  % \begin{frame}[fragile]
  %   \frametitle{Factorization in regular categories}
  %   Let $\mathcal{C}$ be a regular category and $f : X \to Y$.
  %   \pause \\
  %   Then $f$ can be factorized into
  %   \begin{center}
  %     \begin{tikzcd}
  %       X \ar[dr, two heads, "e"'] \ar[rr, "f"] & & Y \\
  %       & I \ar[ur, tail, "m"'] &
  %     \end{tikzcd}
  %   \end{center}
  %   where $e$ is a regular epimorphism and $m$ is a monomorphism.
  %   \pause
  %   This factorization is unique up to isomorphism.
  % \end{frame}

  % \begin{frame}
  %   \frametitle{Images of morphisms}
  %   In the factorization $f = m \circ e$ we call the subobject represented by $m$
  %   the \emph{image} of $f$.
  %   \pause \\
  %   \vspace*{1em}
  %   We denote this as either $\mathrm{Im}(f)$ or $\exists_f X$.
  %   \pause \\
  %   \vspace*{1em}
  %   For a subobject $\alpha : A \to X$ we define its image under $f$ as
  %   \[ \exists_f A := \mathrm{Im}(f \circ \alpha) \]
  %   This gives a well defined map $\exists_f : \mathrm{Sub}(X) \to \mathrm{Sub}(Y)$.
  % \end{frame}
  
  % % \begin{frame}
  % %   \frametitle{Relations}
  % %   We can view a subobject $R \rightarrowtail X \times Y$ as a relation
  % %   between "elements" of $X$ and $Y$. We call a relation $R$:
  % %   \pause
  % %   \begin{itemize}
  % %     \item \emph{Total} if $\exists_{\pi_1}R = X$ 
  % %       \pause
  % %       (intuitively this means
  % %       that the set of all $x$, such that there exists a $y$ for which $xRy$
  % %       is equal to $X$).
  % %       \pause
  % %     \item \emph{Functional} if the cannonical morphism 
  % %       $R \times_X R \to X \times Y \times Y$ factors through the inclusion
  % %       $\mathrm{id}_X \times \Delta_Y : X \times Y \to X \times Y \times Y$
  % %       \pause
  % %       (intuitively this means that from $xRy_1$ and $xRy_2$ it follows
  % %       $y_1 = y_2$).
  % %   \end{itemize}

  % % \end{frame}

  % % \begin{frame}
  % %   \frametitle{Graph of a morphism}
  % %   \begin{definition}
  % %     Let $f : X \to Y$ be a morphism in a regualr category $\mathcal{C}$.
  % %     Then the \emph{graph} of $f$ is the subobject
  % %     \[  \graph (f) \to X \times Y \]
  % %     defined as the image of the morphism $\langle \mathrm{id}_X, f \rangle$.
  % %   \end{definition}
  % %   \begin{block}{Note}
  % %     The cannonical morphism $X \to \graph(f)$ is an isomorphism.
  % %   \end{block}
  % % \end{frame}

  % % \begin{frame}
  % %   \frametitle{Graph of a morphism}
  % %   \begin{proposition}
  % %     Let $\mathcal{C}$ be a regular category.
  % %     \begin{enumerate}
  % %       \item[(i)]<2-> The graph of a morphism $f : X \to Y$ is a total and functional
  % %         relation on $X \times Y$. 
  % %       \item[(ii)]<3-> For each total and functional relation $R \rightarrowtail X \times Y$
  % %         there exists a unique morphism $f : X \to Y$ such that $R = \graph(f)$. 
  % %     \end{enumerate}
  % %   \end{proposition}
  % % \end{frame}

  % \begin{frame}
  %   \frametitle{Interpretation and model}
  %   Let $\mathcal{L}(\Sigma)$ be a language. An \emph{interpretation} $M$ in a
  %   category $\mathcal{C}$ consists of:
  %   \begin{itemize}
  %     \vspace{1em}
  %     \item<2-> An object $X^{(M)}$ for each sort $X \in \mathrm{sort}_\Sigma$.
  %     \vspace{1em}
  %     \item<3-> A morphism $f^{(M)} : X_1^{(M)} \times X_n^{(M)} \to Y^{(M)}$
  %       for each function symbol $f : (X_1, \ldots, X_n) \to Y$.
  %     \vspace{1em}
  %     \item<4-> A subobject $R^{(M)} \rightarrowtail X_1^{(M)} \times X_n^{(M)}$
  %       for each relation symbol $R : (X_1, \ldots, X_n)$.
  %   \end{itemize}
  % \end{frame}

  % \begin{frame}
  %   We extend the interpretation to all terms and formulas of $\mathcal{L}(\Sigma)$.
  %   \pause \\
  %   \vspace{1em}
  %   To each term $t : Y$ in the context $\bar{z} : \bar{Z}$ we assign a morphism 
  %   $t(\bar{z}) : \bar{Z}^{(M)} \to Y^{(M)}$.
  %   \pause \\
  %   \vspace{1em}
  %   To each formula $\varphi$ in the context $\bar{z} : \bar{Z}$ we assign a subobject
  %   $\{ \bar{z} \mid \varphi \}^{(M)} \rightarrowtail \bar{Z}^{(M)}$.
  % \end{frame}

  % \begin{frame}
  %   \begin{enumerate}
  %     \item<1-> If $x$ is a variable of sort $X$ then $x(\bar{z})^{(M)}$ is defined as
  %       the composition $\bar{Z}^{(M)} \xrightarrow{\pi} X^{(M)} \xrightarrow{\mathrm{id}} X^{(M)}$.
  %       \vspace{1em}
  %     \item<2-> If $f : (X_1, \ldots, X_n) \to Y$ is a function symbol and $t_i$ are terms
  %       of sort $X_i$ then $f(t_1, \ldots, t_n)^{(M)}$ in the context $\bar{z} : \bar{Z}$ 
  %       is defined as the composition
  %       $ \bar{Z}^{(M)} \xrightarrow{\langle t_1^{(M)}, \ldots, t_n^{(M)} \rangle} \bar{X}^{(M)} 
  %       \xrightarrow{f^{(M)}} Y^{(M)}$.
  %   \end{enumerate}
  % \end{frame}

  % \begin{frame}[fragile]
  %   \begin{enumerate}
  %     \setcounter{enumi}{2}
  %     \item<1-> To the formula $\bar{z} : \bar{Z} \mid t_1 = t_2$ we assign a subobject
  %       $\{\bar{z} \mid t_1 = t_2 \}^{(M)}$ defined as the equaliser
  %       \begin{center}
  %         \begin{tikzcd}
  %           \{\bar{z} \mid t_1 = t_2 \}^{(M)} \ar[r] & \bar{Z}^{(M)} \ar[r, shift left, "t_1(\bar{z})^{(M)}"] \ar[r, shift right, "t_2(\bar{z})^{(M)}"'] &
  %           X^{(M)}
  %         \end{tikzcd}
  %       \end{center}
  %     \item<2-> For a relation symbol $R : \bar{X}$ we define $\{ \bar{z} \mid R(t_1, \ldots, t_n)\}^{(M)}$
  %       via the pullback
  %       \begin{center}
  %         \begin{tikzcd}
  %           \{\bar{z} \mid R(t_1, \ldots, t_n)\}^{(M)} \ar[d] \ar[r, hook] &
  %           \bar{Z}^{(M)} \ar[d, "\langle {t_1^{(M)}, \ldots ,t_n^{(M)}} \rangle"] \\
  %           R^{(M)} \ar[r, hook] & \bar{X}^{(M)}
  %         \end{tikzcd}
  %       \end{center}
  %   \end{enumerate}
  % \end{frame}

  % \begin{frame}
  %   \begin{enumerate}
  %     \setcounter{enumi}{4}
  %     \item<1-> $\{ \bar{z} \mid \top \}^{(M)}$ is equal $\bar{Z}^{(M)}$.
  %     \vspace{1em}
  %     \item<2-> $\{\bar{z} \mid \varphi \wedge \psi\}^{(M)} = \{\bar{z} \mid
  %       \varphi\}^{(M)} \wedge \{\bar{z} \mid \psi\}^{(M)}$.
  %     \vspace{1em}
  %     \item<3-> $\{\bar{z} \mid \exists x \varphi\}^{(M)} =
  %       \exists_{\pi}\{(x, \bar{z}) \mid \varphi\}^{(M)}$, where
  %       $\pi$ is the projection
  %       $X^{(M)} \times \bar{Z}^{(M)} \xrightarrow{\pi} \bar{Z}^{(M)}$.
  %   \end{enumerate}
  % \end{frame}

  % \begin{frame}
  %   \frametitle{Model}
  %   \begin{definition}
  %     An interpretation $M$ is called a \emph{model for the sequent}
  %     $\bar{x}: \bar{X} \mid \varphi \vdash \psi$ 
  %     \pause
  %     if 
  %     $$\{ \bar{x} \mid \varphi \}^{(M)} \leq \{\bar{x} \mid \psi \}^{(M)}$$
  %     as subobjects of $\bar{X}$.  \pause
  %     We denote this as
  %     \( M \models \varphi \vdash \psi \).
  %   \end{definition}
  %   \pause
  %   \begin{definition}
  %     An interpretation $M$ is a \emph{model of the theory} $T$ if it is a model
  %     for each sequent $\bar{x}: \bar{X} \mid \varphi \vdash \psi$ in $T$.
  %   \end{definition}
  % \end{frame}

  \begin{frame}
    \frametitle{Gameplan}
    \begin{itemize}
      \item Define regular logic \checkmark
      \item Define regular categories \checkmark
      \item Examples \checkmark
      \item Regular epimorphisms \checkmark
      \item Interpretation \& model \checkmark
      \item Soundness
      \item Category of models
      \item Internal logic
      \item Syntactic category
    \end{itemize}
  \end{frame}

  \begin{frame}
    \frametitle{Recap of part I}
    \emph{Regular logic}: terms and formulas in the language $\mathcal{L}(\Sigma)$ 
    \begin{itemize}
      \item<2-> countable set of sorts $X_1, X_2, X_3, \ldots$
      \item<3-> variables $x : X$ are terms
      \item<3-> $f(t_1, \ldots, t_n)$ are terms, where $t_i$ are terms and $f : \bar{X} \to Y$
        is a function symbol
      \item<4-> $t_1 = t_2$ is a formula, where $t_1, t_2$ are terms
      \item<5-> $R(t_1, \ldots, t_n)$ is formula, where $t_i$ are terms and $R : \bar{X}$
         is a relation symbol
      \item<6-> $\varphi \land \psi$ and $\exists (x : X) \varphi$ are formulas, where $\varphi$
        and $\psi$ are formulas
    \end{itemize}
  \end{frame}

  \begin{frame}
    \frametitle{Recap of part I}
    A \emph{sequent} in the context $\Gamma$ is written as
    \[ \Gamma \mid \varphi \vdash \psi \]
    for formulas $\varphi$ and $\psi$.
    \pause
    We define derivations via inference rules of the form
    \begin{prooftree}
      \AxiomC{$\Gamma_1 \mid \varphi_1 \vdash \psi_1$} \AxiomC{$\ldots$} \AxiomC{$\Gamma_n \mid \varphi_n \vdash \psi_n$}
      \TrinaryInfC{$\Delta \mid \theta \vdash \sigma$}
    \end{prooftree}
  \end{frame}

  \begin{frame}
    \frametitle{Structural rules}
      \begin{enumerate}
      \item
        \begin{prooftree}
          \AxiomC{} \UnaryInfC{$\Gamma \mid \varphi \vdash \varphi$}
        \end{prooftree}
        \note[item]{reflexivity}
      \item
        \begin{prooftree}
          \AxiomC{$\Gamma \mid \varphi \vdash \psi$} \AxiomC{$\Gamma \mid \psi \vdash \rho$}
          \BinaryInfC{$\Gamma \mid \varphi \vdash r$}
        \end{prooftree}
        \note[item]{transitivity}
      \item
        \begin{prooftree}
          \AxiomC{$\Gamma \mid \varphi \vdash \psi$}
          \UnaryInfC{$\Gamma, y:Y \mid \varphi \vdash \psi$}
        \end{prooftree}
        \note[item]{superflous variable}
      \item
        \begin{prooftree}
          \AxiomC{$\Gamma, y:B \mid \varphi \vdash \psi$}
          \AxiomC{$\Gamma \mid b : B$}
          \BinaryInfC{$\Gamma \mid \varphi[b/y] \vdash \psi[b/y]$}
        \end{prooftree}
        \note[item]{substitution}
      \end{enumerate}

  \end{frame}


  \begin{frame}
  \frametitle{Rules of logic}

      \begin{enumerate}\addtolength{\itemsep}{0.7\baselineskip}
    \item
      \begin{prooftree}
        \AxiomC{} \UnaryInfC{$\Gamma \mid \varphi \vdash \top$}
      \end{prooftree}
    \item
      \begin{prooftree}
        \AxiomC{$\Gamma \mid \rho \vdash \varphi$} \AxiomC{$\Gamma \mid \rho \vdash \psi$} \doubleLine
        \BinaryInfC{$\Gamma \mid \rho \vdash \varphi \wedge \psi$}
      \end{prooftree}
    \item
      \begin{prooftree}
        \AxiomC{$\Gamma \mid \exists (y:Y) \psi \vdash \varphi$} \doubleLine
        \UnaryInfC{$\Gamma, y:Y \mid \psi \vdash \varphi$}
      \end{prooftree}
    \end{enumerate}
  \end{frame}

  \begin{frame}
    \frametitle{Rules of equality}
    
    \begin{enumerate}\addtolength{\itemsep}{0.7\baselineskip}
    \item $x:X \mid \top \vdash x = x$
    \item
      $x_1:X,x_2:X \mid x_1 = x_2 \vdash x_2 = x_1$
    \item
      $x_1,x_2,x_3:X \mid x_1 = x_2 \wedge x_2 = x_3 \vdash x_1 = x_3$
    \item
      $\bar{x},\bar{y}:\bar{X} \mid \bar{x} = \bar{y} \vdash
      f(\bar{x}) = f(\bar{y})$
    \item
      $\bar{x},\bar{y} : \bar{X} \mid \bar{x} = \bar{y} \wedge R(\bar{x}) \vdash
      \mathrm{R}(\bar{y})$, where
      $R : \bar{X}$
    \end{enumerate}
  \end{frame}

  \begin{frame}
    \frametitle{Recap of part I}

    A \emph{theory} $T$ is a set of sequents.
    \pause
    We define a derivation relation in a theory $T$
    as all derivations from the above rules together with an axtiom for each sequent of the theory.
  \end{frame}

  \begin{frame}[fragile]
    \frametitle{Recap of part I}
    \emph{Regular categories}: categories with a nice algebra of {\bf subobjects} and a notion of
    {\bf image factorization}.
    $\mathbf{Sub}(X)$:
    \begin{center}
      \begin{tikzcd}[column sep=small, row sep=small]
        A \ar[dr, tail, "\alpha"'] \ar[rr] & & B \ar[dl, tail, "\beta"] \\
        & X &
      \end{tikzcd}
    \end{center}

    {\bf Meet} of two subobjects is their pullback:
    \begin{center}
      \begin{tikzcd}
        A \land B \ar[d, tail] \ar[r, tail] & B \ar[d, tail, "\beta"] \\
        A \ar[r, tail, "\alpha"] & X
      \end{tikzcd}
    \end{center}
  \end{frame}

  \begin{frame}[fragile]
    \frametitle{Recap of part I}
    In a regular category we can factorize morphisms
    \begin{equation*}
      \begin{tikzcd}
        X \ar[dr, two heads, "e"] \ar[rr, "f"] && Y \\
        & \exists_f X \ar[ur, tail, "m"]
      \end{tikzcd}
    \end{equation*}
    into a regular epimorphism $e$ and monomorphism $m$.
    \pause
    A regular epi is a coequalizer of some pair of arrows
    \begin{equation*}
      \begin{tikzcd}
        Z \ar[r, shift left] \ar[r, shift right] & X \ar[r, two heads, "e"] & \exists_f X
      \end{tikzcd}
    \end{equation*}
  \end{frame}

  \begin{frame}[fragile]
    \frametitle{Recap of part I}

    We define the image of a subobject $A \in Sub(X)$ as the image of the composition
    \begin{equation*}
      \begin{tikzcd}
        A \ar[r, tail, "\alpha"] \ar[d, two heads] &  X \ar[d, "f"] \\
        \exists_f A \ar[r, tail] & Y
      \end{tikzcd}
    \end{equation*}
    This gives us a map $\exists_f : Sub(X) \to Sub(Y)$
  \end{frame}

  \begin{frame}
    \frametitle{Recap of part I}
    An \emph{interpretation} $M$ in a
    category $\mathcal{C}$ consists of:
    \begin{itemize}
      \vspace{1em}
      \item<2-> An object $X^{(M)}$ for each sort $X \in \mathrm{sort}_\Sigma$.
      \vspace{1em}
      \item<3-> A morphism $f^{(M)} : X_1^{(M)} \times X_n^{(M)} \to Y^{(M)}$
        for each function symbol $f : (X_1, \ldots, X_n) \to Y$.
      \vspace{1em}
      \item<4-> A subobject $R^{(M)} \rightarrowtail X_1^{(M)} \times X_n^{(M)}$
        for each relation symbol $R : (X_1, \ldots, X_n)$.
    \end{itemize}
  \end{frame}

  \begin{frame}
    We extend the interpretation to all terms and formulas of $\mathcal{L}(\Sigma)$.
    \pause \\
    \vspace{1em}
    To each term $t : Y$ in the context $\bar{z} : \bar{Z}$ we assign a morphism 
    $t(\bar{z}) : \bar{Z}^{(M)} \to Y^{(M)}$.
    \pause \\
    \vspace{1em}
    To each formula $\varphi$ in the context $\bar{z} : \bar{Z}$ we assign a subobject
    $\{ \bar{z} \mid \varphi \}^{(M)} \rightarrowtail \bar{Z}^{(M)}$.
  \end{frame}

  \begin{frame}
    \begin{enumerate}
      \item<1-> If $x$ is a variable of sort $X$ then $x(\bar{z})^{(M)}$ is defined as
        the composition $\bar{Z}^{(M)} \xrightarrow{\pi} X^{(M)} \xrightarrow{\mathrm{id}} X^{(M)}$.
        \vspace{1em}
      \item<2-> If $f : (X_1, \ldots, X_n) \to Y$ is a function symbol and $t_i$ are terms
        of sort $X_i$ then $f(t_1, \ldots, t_n)^{(M)}$ in the context $\bar{z} : \bar{Z}$ 
        is defined as the composition
        $ \bar{Z}^{(M)} \xrightarrow{\langle t_1^{(M)}, \ldots, t_n^{(M)} \rangle} \bar{X}^{(M)} 
        \xrightarrow{f^{(M)}} Y^{(M)}$.
    \end{enumerate}
  \end{frame}

  \begin{frame}[fragile]
    \begin{enumerate}
      \setcounter{enumi}{2}
      \item<1-> To the formula $\bar{z} : \bar{Z} \mid t_1 = t_2$ we assign a subobject
        $\{\bar{z} \mid t_1 = t_2 \}^{(M)}$ defined as the equaliser
        \begin{center}
          \begin{tikzcd}
            \{\bar{z} \mid t_1 = t_2 \}^{(M)} \ar[r] & \bar{Z}^{(M)} \ar[r, shift left, "t_1(\bar{z})^{(M)}"] \ar[r, shift right, "t_2(\bar{z})^{(M)}"'] &
            X^{(M)}
          \end{tikzcd}
        \end{center}
      \item<2-> For a relation symbol $R : \bar{X}$ we define $\{ \bar{z} \mid R(t_1, \ldots, t_n)\}^{(M)}$
        via the pullback
        \begin{center}
          \begin{tikzcd}
            \{\bar{z} \mid R(t_1, \ldots, t_n)\}^{(M)} \ar[d] \ar[r, hook] &
            \bar{Z}^{(M)} \ar[d, "\langle {t_1^{(M)}, \ldots ,t_n^{(M)}} \rangle"] \\
            R^{(M)} \ar[r, hook] & \bar{X}^{(M)}
          \end{tikzcd}
        \end{center}
    \end{enumerate}
  \end{frame}

  \begin{frame}[fragile]
    \begin{enumerate}
      \setcounter{enumi}{4}
      \item<1-> $\{ \bar{z} \mid \top \}^{(M)}$ is equal $\bar{Z}^{(M)}$.
      \vspace{1em}
      \item<2-> $\{\bar{z} \mid \varphi \wedge \psi\}^{(M)} = \{\bar{z} \mid
        \varphi\}^{(M)} \wedge \{\bar{z} \mid \psi\}^{(M)}$.
      \vspace{1em}
      \item<3-> $\{\bar{z} \mid \exists x \varphi\}^{(M)} =
        \exists_{\pi}\{(\bar{z}, x) \mid \varphi\}^{(M)}$, where
        $\pi$ is the projection
        $\bar{Z}^{(M)}  \times X^{(M)} \xrightarrow{\pi} \bar{Z}^{(M)}$.
        \begin{equation*}
          \begin{tikzcd}
            \{ (\bar{z}, x) \mid \varphi\}^{(M)} \ar[d, two heads] \ar[r, tail] & \bar{Z}^{(M)} \times X^{(M)} \ar[d, "\pi"] \\
            \{\bar{z} \mid \exists(x : X) \varphi\}^{(M)} \ar[r, tail] & \bar{Z}^{(M)}
          \end{tikzcd}
        \end{equation*}
    \end{enumerate}
  \end{frame}

  \begin{frame}
    \frametitle{Model}
    An interpretation $M$ is called a \emph{model for the sequent}
    $\bar{x}: \bar{X} \mid \varphi \vdash \psi$ 
    if 
    $$\{ \bar{x} \mid \varphi \}^{(M)} \leq \{\bar{x} \mid \psi \}^{(M)}$$
    as subobjects of $\bar{X}$.
    We denote this as
    \( M \models \varphi \vdash \psi \).
    \pause \\
    An interpretation $M$ is a \emph{model of the theory} $T$ if it is a model
    for each sequent in $T$.
  \end{frame}

  \begin{frame}
    \frametitle{Soundness}
    \begin{theorem}
      Let $T$ be a regular theory and $M$ a model of $T$ in a regular category
      $\mathcal{C}$. \pause
      If there is a derivation
      $\bar{x} : \bar{X} \mid \varphi \vdash \psi$ in the theory $T$, then
      $M \models \varphi \vdash \psi$.
    \end{theorem}
    \pause
    \begin{block}{Proof}
      By induction on the structure of derivations.
    \end{block}
  \end{frame}

  \begin{frame}
    \frametitle{Proof of soundness}
    \begin{itemize}
      \item For an axiom $\Gamma \mid \varphi \vdash \psi$ we have by definition that
        \[ \{\Gamma \mid \varphi \}^{(M)} \leq \{\Gamma \mid \psi \}^{(M)} \]
      \item Reflexivity $\Gamma \mid \varphi \vdash \varphi$ holds via the identity
        \[ \{\Gamma \mid \varphi \}^{(M)} \overset{\mathrm{id}}{\rightarrowtail} \{\Gamma \mid \varphi\}^{(M)} \]
    \end{itemize}
  \end{frame}

  \begin{frame}
    \frametitle{Proof of soundness}
    \begin{itemize}
      \item Transitivity: by induction we have 
        $$\{\bar{x}:\bar{X} \mid \varphi \}^{(M)} \leq \{\bar{x}:\bar{X} \mid \psi \}^{(M)}$$ and
        $$\{\bar{x}:\bar{X} \mid \psi \}^{(M)} \leq \{\bar{x}:\bar{X} \mid \rho \}^{(M)}$$
        Then by transitivity of $\leq$ in $\mathrm{Sub}(\bar{X}^{(M)})$ we get
        \[\{\bar{x}:\bar{X} \mid \varphi \}^{(M)} \leq \{\bar{x}:\bar{X} \mid \rho \}^{(M)}\]
    \end{itemize}
  \end{frame}

  \begin{frame}[fragile]
        \begin{prooftree}
          \AxiomC{$\Gamma \mid \varphi \vdash \psi$}
          \UnaryInfC{$\Gamma, y:Y \mid \varphi \vdash \psi$}
        \end{prooftree}
        We prove that
        \[\{(\bar{z},y) \mid \varphi \}^{(M)} = \pi^{-1}\{\bar{z} \mid \varphi \}^{(M)}\]
        by induction on the structure of $\varphi$.
        \begin{equation*}
          \begin{tikzcd}[column sep=small, row sep=tiny]
            \{(\bar{z},y) \mid \varphi \} \ar[dd] \ar[dr, tail] \ar[rr, dashed] & & \{(\bar{z},y) \mid \psi \} \ar[dl, tail] \ar[dd]\\
            & \bar{Z} \times Y  & \\
            \{\bar{z} \mid \varphi \} \ar[dr, tail] \ar[rr, tail] & & \{\bar{z} \mid \psi \} \ar[dl, tail] \\
            & \bar{Z} \ar[uu, leftarrow, pos=0.8, crossing over, "\pi"'] &
          \end{tikzcd}
        \end{equation*}
  \end{frame}

  \begin{frame}
    \frametitle{Proof of soundness}
    \begin{itemize}
      \item Substitution 
        \begin{prooftree}
          \AxiomC{$\Gamma, y:B \mid \varphi \vdash \psi$}
          \AxiomC{$\Gamma \mid b : B$}
          \BinaryInfC{$\Gamma \mid \varphi[b/y] \vdash \psi[b/y]$}
        \end{prooftree}
        follows from another lemma that we prove by induction on the structure of formulas.
      \item $\Gamma \mid \varphi \vdash \top$
        By definition $\{\bar{x}:\bar{X} \mid \top\}^{(M)} = \bar{X}^{(M)}$
      \item Conjunction. We defined
        $\{\bar{z} \mid \varphi \wedge \psi\}^{(M)} = \{\bar{z} \mid
                \varphi\}^{(M)} \wedge \{\bar{z} \mid \psi\}^{(M)}$.
        Use induction and properties of pullbacks.
    \end{itemize}
  \end{frame}

  \begin{frame}
    \frametitle{Proof of soundness}
    \begin{prooftree}
      \AxiomC{$\Gamma \mid \exists (y:Y) \psi \vdash \varphi$} \doubleLine
      \UnaryInfC{$\Gamma, y:Y \mid \psi \vdash \varphi$}
    \end{prooftree}
    We again use the lemma for superflous variables, induction and interpretation of $\exists$.
    \pause \\
    \vspace{1em}
    The rest are mostly diagram chasing and induction.
    \\
    \hfill$\square$
  \end{frame}

  \begin{frame}
    \frametitle{Category of models}
    Let $T$ be a regular theory. \pause For any regular category $\mathcal{C}$ we can form
    the category of models of $T$ in $\mathcal{C}$, which we denote as
    $\mathbf{Mod}(T, \mathcal{C})$.
    \pause \\
    The morphisms in $\mathbf{Mod}(T, \mathcal{C})$ are families of morphisms
    $$\{h_X : X^{(M)} \to X^{(N)} \}_{X \in \mathrm{sort}_{\Sigma}},$$
    \pause
    which commute with the interpretations of function and relation symbols of
    $\mathcal{L}(\Sigma)$. \pause We extend this to all terms and formulas by induction.

  \end{frame}

  \begin{frame}
    \frametitle{Internal logic}
    Let $\mathcal{C}$ be a regular category. \pause
    We assign to it a regular signature
    $\Sigma_{\mathcal{C}}$ and language $\mathcal{L}(\Sigma_{\mathcal{C}})$.
    \pause
    \vspace{1em}
    \begin{itemize}
      \item The sorts of $\Sigma_{\mathcal{C}}$ are the objects of $\mathcal{C}$.
      \pause
      \vspace{1em}
      \item For each morphism $f : \bar{X} \to Y$ we add a function symbol
        $f_{\bar{X},Y} : \bar{X} \to Y$ to $\mathcal{L}(\Sigma_{\mathcal{C}})$.
      \pause
      \vspace{1em}
      \item For each subobject $R \rightarrowtail \bar{X}$ we add a relation symbol
        $R_{\bar{X}} : \bar{X}$ to $\mathcal{L}(\Sigma_{\mathcal{C}})$.
    \end{itemize}
  \end{frame}

  \begin{frame}
    \frametitle{Internal logic}
    $\mathcal{L}(\Sigma_{\mathcal{C}})$ has a cannonical interpretation
    $I^{\mathcal{C}}$ in $\mathcal{C}$. We abuse notation and write $\mathcal{C}$
    instead of $I^{\mathcal{C}}$.
    \vspace{1em}
    \pause \\
    The theory $T_{\mathcal{C}}$ is defined as the theory of this interpretation\pause ,
    i.e. the set of all sequents $\Gamma \mid \varphi \vdash \psi$ which hold for
    the interpretation $I^{\mathcal{C}}$.
    \vspace{1em}
    \pause \\
    This allows us to describe the categorical structure of a regular category in
    terms of regular logic.
  \end{frame}

  \begin{frame}
    \begin{lemma}
      Let $\mathcal{C}$ be a regular category.
      \begin{itemize}
      \item<2-> Let $X \xrightarrow{f} Y \xrightarrow{g} Z$ and
        $X \xrightarrow{h} Z$.  Then $h = g \circ f$
        if and only if
        \[{\mathcal{C} \models (x:X \mid  h(x) = g(f(x)))}.\]
      \item<3-> A morphism $m : X \to Y$ is mono exactly when
        \[\mathcal{C} \models (x_1,x_2:X \mid m(x_1) = m(x_2) \vdash x_1 = x_2). \]
      \item<4-> A morphism $f : X \to Y$ is a regular epimorphism exactly when
        \[ \mathcal{C} \models (y:Y \mid \exists (x:X) f(x) = y). \]
      \end{itemize}
    \end{lemma}
  \end{frame}

  \begin{frame}
    \begin{lemma}
      Let $\mathcal{C}$ be a regular category.
      \begin{itemize}
        \item An object $X \in \mathcal{C}$ is terminal if and only if
      \end{itemize}
      \[ \mathcal{C} \models (x_1,x_2:X \mid x_1 = x_2) \quad \text{and} \quad
      {\mathcal{C} \models \exists (x:X) (x=x)}.\]
      \pause
      \begin{itemize}
        \item The morphisms $f : Z \to X$ and $g : Z \to Y$ make $Z$ into
        the product of  $X$ and $Y$ if and only if
      \end{itemize}
      \[\mathcal{C} \models (z_1,z_2 \mid f(z_1) = f(z_2) \wedge
        g(z_1) = g(z_2) \vdash z_1 = z_2)\] and
      \[\mathcal{C} \models (x:X,y:Y \mid \exists (z:Z)(f(z) = x \wedge
        g(z) = y)).\]
    \end{lemma}
  \end{frame}

  \begin{frame}
    \begin{corollary}
      We can express that a category is regular in terms of its internal language.
    \end{corollary}
    \pause
    \begin{corollary}
      We can express that a functor between two regular categories is regular in
      terms of the internal languages of the two categories.
    \end{corollary}
  \end{frame}

  \begin{frame}
    \frametitle{The syntactic category}
    Let $T$ be a regular theory. We define its \emph{syntactic category} $\mathcal{R}(T)$
    as:
    \pause
    \begin{itemize}
      \item The objects are equivalence classes of formulas in a context, i.e.
        $\Gamma \mid \varphi$. \pause \\
        Two formulas $\Gamma \mid \varphi$ and $\Gamma \mid \psi$
        represent the same object, when we can derive $\Gamma \mid \varphi \vdash \psi$
        and $\Gamma \mid \psi \vdash \varphi$ in $T$. \pause \\
        Denote the equivalence
        class as $\{ \Gamma \mid \varphi \}$.
    \end{itemize}
  \end{frame}

  \begin{frame}[fragile]
    \begin{itemize}
      \item A morphism 
        $\{\bar{x},\bar{y} \mid \gamma(\bar{x}, \bar{y}) \} : \{\bar{x} \mid \varphi \} \to \{\bar{y} \mid \psi \}$
        is a functional relation $\gamma$ in $T$. \pause 
        This means we have derivations:
        \begin{itemize}
          \item  $\bar{x}:\bar{X},\bar{y}:\bar{Y} \mid \gamma(\bar{x},\bar{y})
            \vdash p(\bar{x}) \wedge q(\bar{y})$
          \item $\bar{x}:\bar{X} \mid p(\bar{x}) \vdash \exists \bar{y} \gamma(\bar{x},\bar{y})$
          \item $\bar{x}:\bar{X},\bar{y}_1,\bar{y}_2:\bar{Y} \mid
            \gamma(\bar{x},\bar{y}_1) \wedge \gamma(\bar{x},\bar{y}_2) \vdash \bar{y}_1 = \bar{y}_2$
        \end{itemize}
      \pause
      \item The composition of
        \begin{equation*}
          \begin{tikzcd}
            \lbrace \bar{x} : \bar{X} \mid p \rbrace \ar[r, "{\lbrace \varphi \rbrace}"] & 
            \lbrace \bar{y}:\bar{Y} \mid q \rbrace \ar[r, "{\lbrace \psi \rbrace}"] & 
            \lbrace \bar{z}:\bar{Z} \mid r \rbrace
          \end{tikzcd}
        \end{equation*}
        is given by the formula
        \[ \bar{x}:\bar{X}, \bar{z}:\bar{Z} \mid \exists (\bar{y}:\bar{Y}) (\varphi(\bar{x},\bar{y}) \wedge
          \psi(\bar{y},\bar{z})).\]
    \end{itemize}
  \end{frame}

  \begin{frame}
    \begin{proposition}
      The syntactic category $\mathcal{R}(T)$ is a regular category.
    \end{proposition}
    \pause
    It contains a cannonical interpretation $\mathcal{U}$ of the language 
    $\mathcal{L}(\Sigma)$:
    \pause
    \begin{itemize}
      \item $X^{(\mathcal{U})} = \{ x : X \mid \top \}$, where $x : X$ is a variable.
      \pause
      \item $f^{(\mathcal{U})} = \{ \bar{x}:\bar{X}, y : Y \mid f(\bar{x}) = y\}$, where
        $f : \bar{X} \to Y$ is a function symbol.
      \pause
      \item $R^{(\mathcal{U})} = \{ \bar{x} : \bar{X} \mid R(\bar{x})\}$, where
        $R : \bar{X}$ is a relation symbol.
    \end{itemize}
    \pause
    Extend this to all terms and formulas with induction.
  \end{frame}

  \begin{frame}
    \begin{theorem}
      The cannonical interpretation $\mathcal{U}$ is a model of the theory $T$
      in $\mathcal{R}(T)$. \pause
      The rules of inference of
      regular logic are complete with regard to interpretations in small regular
      categories.
    \end{theorem}
    \pause
    \begin{theorem}
      For any regular category $\mathcal{C}$ there is an equivalence of categories
      \[ \mathbf{Mod}(T, \mathcal{C}) \simeq \mathbf{RegCat}(\mathcal{R}(T), \mathcal{C}) \]
    \end{theorem}
  \end{frame}

\end{document}

